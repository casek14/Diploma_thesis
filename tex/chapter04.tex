\section{Edge cloud}
 Edge computing, neboli Edge cloud, je nový směr vývoje cloud computingu. Cílem Edge computingu je přinést výpočetní zdroje blíže k místu, kde jsou potřeba. Cloud native aplikace mohou být využity pro Edge cloud. Mikroslužby zabalené v kontejnerech se všemi závislostmi mohou být použity pro nasazení aplikací na Edge cloud. Kontejnery nejsou náročné na výpočetní zdroje, díky tomu není potřeba nasazovat do Edge cloudu mnoho serverů, stačí pár serverů, které jsou optimalizované pro běh \linebreak kontejnerů. S využitím orchestrátoru kontejnerů Kubernetes je možné nasadit stejnou aplikaci ve více Edge cloudech. Aplikace vypadá pro všechny Edge stejně a z tohoto důvodu se dobře udržuje a stejným způsobem se budou spravovat i jednotlivé k8s clustery napříč Edge cloudy. K8s dále nabízí rychlé doručování nových verzí aplikací \linebreak na jednotlivé Edge cloudy a dovoluje tak reagovat na nové požadavky.\par
 Cloud computing využívá centralizovaný model. Všechny zdroje jsou situovány \linebreak v datovém centru s velkou kapacitou výpočetních zdrojů, ke kterému jednotlivé služby přistupují. Edge computing je na druhé straně distribuovaný model s omezeným objemem zdrojů a posouvá se blíže ke službám, které ho využívají. Centralizovaný cloud computing model není vhodný pro IoT (Internet of things) zařízení, které v posledních letech zažívají velký rozvoj a představují nové výzvy, které musí společnosti řešit. Jak uvádí \cite{buyya}, mezi otázky kterými se společnosti v souvislosti s Edge cloudem a IoT zařízeními musí zabývat patří šířka pásma pro přenos dat, latence neboli zpoždění, stabilita, bezpečnost a omezené zdroje IoT zařízení. IoT zařízení produkující velké objemy dat mohou překročit šířku pásma pro dané přenosové médium. Příkladem mohou být \linebreak samořiditelná auta, která mohou produkovat gigabajty dat za sekundu, a přenos takto velkého objemu dat do vzdáleného datového centra zpomaluje proces rozhodování o chování automobilu v dané situaci. Pro spolehlivou a rychlou komunikaci je nutné mít nízkou odezvu. Jedním z příkladů může být rozšířená realita. Pokud si přes brýle pro rozšířenou realitu prohlížíme stroje ve výrobní hale, očekáváme, že informace o objektech, které právě vidíme budou v rozumné době brýlemi zobrazeny. Nízká odezva od serveru, který zpracovává požadavky dovolí nabídnout uživatelům příjemný prožitek z používání brýlí. V obou těchto případech je užitečnější, aby jednotlivá zařízení komunikovala s Edge technologií, která se nachází v bezprostřední blízkosti zařízení, než aby IoT zařízení komunikovalo se vzdáleným datacentrem. Komunikace s lokálním Edge cloudem je také spolehlivější a méně náchylná na výpadku a rušení. IoT zařízení mají omezené výpočetní zdroje a energie pro svůj provoz. Z tohoto důvodu je například energeticky náročné přenášet přes internet data, případně provádět výpočetně náročnější operace. Edge může sloužit jako agregátor dat ze zařízení a na základě jejich analýzy vykonávat definované akce. Rozhodnutí o dalších akcích tedy místo tradičního vzdáleného datacentra vykonává místní Edge cloud. To ovšem neznamená, že tradiční datacentrum ztrácí smysl. Edge může odesílat data právě do vzdáleného datacentra, které působí jako agregátor dat z více Edge cloudů, kde může například docházet k hlubší a detailnější analýze dat z více Edge cloudů, případně dalším akcím které vyžadují použití většího objemu výpočetních zdrojů. Jednotlivá IoT zařízení v sobě nezahrnují dostatečné mechanismy, aby byly schopné \linebreak ochránit sebe sama a také přenášená data. Data jsou do Edge cloudu jsou přenášena nechráněně, ale odtud až do datacentra už putují přes zabezpečené kanály. Tímto přistupem se například minimalizuje zneužití zařízení pro vysílání napadených či chybných dat. \par
     Edge cloud je poměrně nový přístup, který budí zájem společností z oblastí mobilních operátorů, rozšířené reality, autonomních a IoT zařízení. Jedním z projektů, který se zaměřuje na vytvoření technologie pro Edge cloud je projekt Akraino Edge Stack \cite{akraino}. Akraino je open-source projekt, který je součástí Linux Foundation. Mezi iniciátory tohoto projektu patří AT&T a Intel. Projekt je veden jako komunitní projekt několika společností mezi něž patří Dell EMC, Ericsson, Huawei, Juniper Networks, Nokia, Qualcomm, Red Hat a další. Jeho cílem je vylepšení Edge cloudové infrastruktury. Akraino projekt je seskupení technologií, které by měly poskytovat novou úroveň \linebreak flexibility pro rychlé škálování Edge služeb a zajistili spolehlivost služeb, které musí pořád běžet. Akraino komunita se zaměřuje na vytvoření Edge API, vývojových \linebreak nástrojů pro Edge a umožní napojení na cloudová řešení třetích stran. Součástí projektu bude podpora pro vývoj Edge aplikací a vytvoření virtual network function (VNF - virtuální sítové funkce jako jsou např. firewally, DNS servery) ekosystému. \par
         Zatímco projekt Akraino je spíše v začátcích, americká síť fast-food občerstvení Chick-fil-A (dále jen Chick) využívá Edge computing k efektivnímu řízení svých restaurací. Každá restaurace obsahuje tři malé servery na kterých je nasazený Kubernetes orchestrátor. Společnost má okolo 2000 poboček, což znamená 6000 zařízení, která běží v Edge cloudu. Pro nasazení k8s na každé zařízení si Chick vyvinulo vlastní řešení, které poskytuje automatickou instalaci zařízení, odolnost vůči výpadku a vysokou dostupnost. Díky tomu jsou schopni rychle reagovat na vznik nových poboček \cite{chick-1}. Chick využívá Edge kvůli nízké latency, aplikacím, které jsou nezávislé na internetovém připojení, vysoké dostupnosti těchto aplikací, škálovatelnosti infrastruktury a také kvůli platformě, která umožňuje rychlé doručování nových funkcí do produkce. Chick restaurace se snaží vybudovat chytrou kuchyni, která bude poskytovat data o dění v kuchyni. Tato data jsou později použita ke zlepšování jednotlivých procesů, čímž mohou dosáhnout na lepší rozšiřitelnost celého byznysu. Data mohou být například použita pro předpověd objemu jídla, který má být v daném okamžiku připraven tak, aby zákazníci nečekali dlouho a pokaždé dostali čerstvě připravený produkt. Jednotlivé časovače, které rozhodují o tom zda připravená surovina jíž nesmí být podána zákazníkům, fungují automaticky. Po naplnění nádoby čerstvě připravenými surovinami a jejím přesunutím pod kamerou, která načte kód umístěný na nádobě se automaticky podle kódu spustí časovač pro daný typ suroviny. Jednotlivé časy jsou poté vyobrazeny \linebreak na dotykových obrazovkách a při vypršení limitu pro servírování jsou zaměstnanci \linebreak upozorněni na vyhození dané suroviny. A právě data a procesy, které rozhodují o těchto akcích běží v Edge cloudu, který je umístěný přímo v restauraci \cite{chick-3}.
	 Páteří řešení je AWS cloud, který slouží jako autorizační server pro zařízení, sběr dat, centrální monitoring, alerting, tracing a správa instalací. Na straně Edge cloudu Chick spravuje autentikaci, messaging s využitím MQTT a mikroslužby vyřizující HTTP požadavky. Dále zde běží aplikace, které interagují se zařízeními v restauraci. Součástí Edge řešení jsou i modely pro predikci budoucího chování na základě událostí ze zařízení v restauraci. Všechna data na straně Edge jsou potupně přenesena do centrálního uložiště a poté \linebreak z Edge uložiště smazána. V případě výpadku sítě umí Edge data agregovat a po opětovném obnovení spojení je nahrát do centrálního datového centra. Poslední vrstva se stará o připojení jednotlivých zařízení v restauraci. Chick používá Kubernetes a kontejnery pro centrální AWS cloud tak i pro Edge cloudy. Díky tomu mají stejné prostředí pro běh aplikací na obou stranách, což usnadnuje práci vývojařům a pomáhá rychlému dodávání změn \cite{chick-2}.

