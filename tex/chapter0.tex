\chapter{Úvod}
\pagenumbering{arabic}
\setcounter{page}{1}
V současné době dochází na poli návrhu a vývoje počítačových systémů k výraznému posunu. Příchod cloud computingu a později cloud native přístupu transformoval základní principy návrhu a běhu aplikací. Namísto fyzické infrastruktury, vývojáři implementují služby nad sofistikovanou virtualizovanou platformou, která přináší výhody oproti klasickému přístupu. První generace systémů, které běželi v cloudovém \linebreak prostředí se nazývaly “cloud enabled” systémy. Tyto systémy, typické pro cloud computing éru, byly přeneseny do prostředí cloudu, ale byly provozovány velice podobně jako v období před cloud computingem. Z tohoto důvodu aplikace plně nevyužívaly výhody, které jim cloud computing nabízel. Pro překonání těchto problémů se vytvořily nové přístupy a technologie v tvorbě aplikací, souhrně nazývané cloud native \cite{hotcloud}. 
\par
    Cloud native přístup k tvorbě a správě aplikací zasahuje do celého životního cyklu aplikace. Architektura cloud native aplikací je založena na mikroslužbách, které \newline rozdělují aplikace na menší vzájemně nezávislé celky. Tyto celky se lépe rozšiřují, udržují a škálují oproti monolitickým aplikacím. Dalším stavebním blokem cloud native přístupu jsou kontejnery. Kontejnery v sobě zapouzdřují závislosti aplikací a \newline napomáhají přenositelnosti aplikací mezi testovacím a produkčním prostředím. Pro správu cloud native aplikací jsou používány orchestrátory kontejnerů, které přináší jednoduchou a automatizovanou práci s kontejnery. 
	\par
        Cloud native přístup má více oblastí použití. První oblastí jsou aplikace, které \linebreak vyžadují vysokou dostupnost a škálovatelnost. Vysoké dostupnosti může být dosaženo s využitím orchestrátorů kontejnerů, které sledují stav jednotlivých mikroslužeb a v případě výpadku jsou schopné nahradit chybné kontejnery. Architektura mikroslužeb dovoluje škálovat pouze potřebnou část aplikace. S rozvojem Internetu věcí (IoT) se cloud native přístup prosazuje i v prostředí Edge computingu. Cílem Edge computingu je zpracovávat data, co nejblíže k jejich zdroji a umožnit tak jejich rychlé zpracovaní a snížení doby odezvy. Stejná situace panuje i v prostředí multi-cloud. Společnosti mají zájem jednotně spravovat zdroje a využívat služby napříč více poskytovateli cloudových služeb. Společnosti jako Google, Amazon, Red Hat a spousta dalších se snaží přijít s vlastním řešením, které nabídne jednotnou správu zdrojů a aplikací. Z těchto důvodů se tato oblast stále rozvíjí a nabízí tak prostor pro nové technologie.
	\par
	    Cílem této diplomové práce je analyzovat a navrhnout platformu, která bude umožňovat běh distribuovaných aplikací v prostředí multi-cloudu. Řešení práce spočívá ve vytvoření požadavků pro takovou platformu, navrhnutí architektury dané platformy a vytvoření prototypu systému, který bude společně s jejím testováním výstupem této práce. 
	    \par
	        Diplomová práce se v teoretické části zaměřuje na vývoj cloud computingu. V první kapitole je popsaný princip cloud computing. Druhá kapitola navazuje na cloud computing a představuje cloud native computing jako nové cloudové paradigma. V kapitole jsou představeny základní principy a technologie cloud native computingu a v čem se odlišuje od předchozího přístupu. Kapitola dále představuje oblasti využití cloud native přistupu v podobě edge cloudu a hybrid cloudu. V podkapitole hybrid cloud jsou dále rozebrány existující nástroje pro tuto oblast. Poznatky z teoretické části jsou dále využity pro definici požadavků na aplikaci, která umožní běh distribuovaných aplikací. Praktická část práce se dále zabývá návrhem takového systému a implementací prototypu. V poslední části práce jsou uvedeny výsledky z testování vytvořené aplikace, které mají za ukol ověřit funkčnost navržené aplikace. 

