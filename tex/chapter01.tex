\chapter{Úvod}
\pagenumbering{arabic}
\setcounter{page}{1}
Směr Cloud Computingu
Označení Cloud computing je některými odborníky považováno za nové paradigma, někteří dokonce mluví o nové technologii, která umožňuje přístup k výpočetním zdrojům a službám přes internet \cite{bohm2010cloud}. Cloud computing, často označovaný pouze jako cloud, tak dovoluje jednotlivcům, malým firmám a dalším subjektům jednoduchý přístup k výpočetnímu výkonu z pohodlí domova či kanceláře za přijatelnou cenu, bez nutnosti spravovat celou výpočetní infrastrukturu. Uživatelé jsou tak odstíněni od konfigurace serverů a síťových zařízení a služeb samotných. \newline
Podle \cite{cc2011principles} lze na Cloud computing nahlížet jako na využívání elektřiny. Elektřinu využíváme jako službu. Nezajímáme se o to jak se elektřina vyrábí a jak je dodávána do jednotlivých elektrických zásuvek v našem pokoji. My pouze připojíme zařízení do elektrické zásuvky a očekáváme, že například nabije náš telefon či počítač. Pokud tento příklad přeneseme do oblasti IT, znamená to, že uživatelé jsou odstínění od vnitřního fungování nějaké služby nebo použitých technologiích. Jako příklad může být brána služba cloudového uložiště fotografií. Pro uživatele je důležité, že si může fotografie prohlížet  z různých zařízení přes internet odkudkoliv a kdykoliv. Ostatní problémy jako jsou uložení a záloha těchto dat, webové rozhraní pro práci s fotografiemi a další je ponecháno na poskytovateli služby. \newline
Některé instituce, akademičtí pracovníci či IT inženýři vytvořili definice a charakteristiky  popisující Cloud computing. Například Americký národní institut standardů a technologií (NIST) \cite{mel2011nist}, definuje cloud computing následovně:
Cloud computing je model pro pohodlný síťový přístup ke skupině konfigurovatelných výpočetních zdrojů, jako jsou počítačové sítě, servery, datová uložiště, aplikace a služby, které jsou dostupné odkudkoliv a okamžitě na vyžádání. Tyto zdroje mohou být rychle vytvořeny a uvolněny  s vyvinutím minimálního úsilí nebo minimální interakce od provozovatele dané služby. \newline 
Definice ze zdroje \cite{Vaquero-cloud-definition} zmiňuje, že uživatelé platí pouze za zdroje, které skutečně využívají podle sjednaných podmínek. Tento postup se nazývá pay-per-use model. Znamená to žádné pevně stanovené poplatky bez ohledu na to zda zákazník využívá polovinu přidělených zdrojů nebo se využití zdrojů blíží 100 \%. Díky tomu je možné si například vypočítat náklady na provozování služby u různých firem a rozhodnout se tak pro nejvhodnější platformu. \newline
    Armbrust a jeho kolegové z Kalifornské Univerzity v Berkeley \cite{Ambrust2009}, shrnují cloud computing do 3 bodů. Prvním bodem je zdání, že uživatel má k dispozici nekonečný objem výpočetních zdrojů. Z tohoto důvodu není potřeba dopředu plánovat zda bude k dispozici dostatek zdrojů, zdroje jsou dynamicky přidávány a odebírány na požádání. Druhým bodem je skutečnost, že není potřebné uzavírat žádné předběžné závazky ze strany uživatele. Toto dovoluje firmám začít s malým počtem zdrojů a postupně navyšovat zdroje s rostoucími potřebami. Třetím a posledním bodem je zde platit za využívané zdroje ve velice krátkém horizontu, např. hodin nebo dní a poté je uvolnit podle potřeb.

