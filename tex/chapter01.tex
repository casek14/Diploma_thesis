\chapter{Cloud computing}
Označení cloud computing se začalo objevovat okolo roku 2008\cite{ZHANG}. Někteří odborníci považovali cloud computing model za nové paradigma, někteří dokonce mluvili o nové technologii, která umožňuje přístup k výpočetním zdrojům a službám přes internet \cite{bohm2010cloud}. Cloud computing, často také označovaný pouze jako cloud, tak dovoluje jednotlivcům, malým firmám a dalším subjektům jednoduchý přístup k výpočetnímu výkonu z pohodlí domova či kanceláře za přijatelnou cenu, bez nutnosti nakupovat, obměňovat \linebreak a spravovat celou výpočetní infrastrukturu. Uživatelé jsou tak odstíněni od konfigurace serverů, síťových zařízení a služeb samotných. \par
Podle \cite{cc2011principles} lze na cloud computing nahlížet jako na využívání elektřiny. Elektřinu využíváme jako službu. Nezajímáme se o to jak se elektřina vyrábí a jak je dodávána do jednotlivých elektrických zásuvek v našem pokoji. My pouze připojíme zařízení \linebreak do elektrické zásuvky a očekáváme, že například nabije náš telefon či počítač. Pokud tento příklad přeneseme do oblasti IT, znamená to, že uživatelé jsou odstínění \linebreak od vnitřního fungování nějaké služby nebo použitých technologiích. Jako příklad může být brána služba cloudového uložiště fotografií. Pro uživatele je důležité, že si může fotografie prohlížet  z různých zařízení přes internet odkudkoliv a kdykoliv. Ostatní problémy jako jsou uložení a záloha těchto dat, webové rozhraní pro práci s fotografiemi a další je ponecháno na poskytovateli služby. \par 
Některé instituce, akademičtí pracovníci či IT inženýři vytvořili definice a charakteristiky  popisující cloud computing. Například Americký národní institut standardů a technologií (NIST) \cite{mel2011nist}, definuje cloud computing následovně.
Cloud computing je model pro pohodlný síťový přístup ke skupině konfigurovatelných výpočetních zdrojů, jako jsou počítačové sítě, servery, datová uložiště, aplikace a služby, které jsou dostupné odkudkoliv a okamžitě na vyžádání. Tyto zdroje mohou být rychle vytvořeny a uvolněny  s vyvinutím minimálního úsilí nebo minimální interakce od provozovatele dané služby. \par
Definice ze zdroje \cite{Vaquero-cloud-definition} zmiňuje, že uživatelé platí pouze za zdroje, které skutečně využívají podle sjednaných podmínek. Tento postup se nazývá pay-per-use model. Znamená to žádné pevně stanovené poplatky bez ohledu na to zda zákazník využívá polovinu přidělených zdrojů nebo se využití zdrojů blíží 100\%. Díky tomu je možné si například vypočítat náklady na provozování služby u různých firem a rozhodnout se tak pro nejvhodnější platformu. \par
    Armbrust a jeho kolegové z Kalifornské Univerzity v Berkeley \cite{Ambrust2009}, shrnují cloud computing do 3 bodů. Prvním bodem, je zdání, že uživatel má k dispozici nekonečný objem výpočetních zdrojů. Z tohoto důvodu není potřeba dopředu plánovat zda bude k dispozici dostatek zdrojů. Zdroje jsou dynamicky přidávány a odebírány na požádání. Druhým bodem je skutečnost, že není potřebné uzavírat žádné předběžné závazky ze strany uživatele. Toto dovoluje firmám začít s malým počtem zdrojů a postupně navyšovat zdroje s rostoucími potřebami. Třetím a posledním bodem je zde platit \linebreak za využívané zdroje ve velice krátkém horizontu, např. hodin nebo dní a poté je uvolnit podle potřeb. \par

    Z výše zmíněných definic vyplývá, že cloud computing umožňuje vzdálený přístup k výpočetním zdrojům. Uživatelé tak mohou přistupovat k aplikacím a datům  \linebreak z různých zařízení a lokací. K tomu využívají jediný účet a na základě definovaných práv mají přístup do různých částí systému. Dalším plusem je jednoduché přidání a odebrání zdrojů podle potřeby, s tím spojené placení pouze za skutečně využívané zdroje a jenom po dobu používání. Tato vlastnost cloudu je velký posun. Společnosti nemusí dlouhodobě plánovat nákup a výměnu hardwaru. Využívají zdroje, které \linebreak aktuálně potřebují. Náklady za zdroje se odvíjejí od momentálního vytížení. Pokud je potřeba zvýšit počet zdrojů, stane se tak automaticky podle předem připraveného scénáře. Například vytvoření dalších webových serverů pro rozložení zátěže a \linebreak odbavení uživatelů při špičce. Poté zase vypnutí nevyužitých zdrojů v době mimo špičku.  \linebreak Spolehlivost a vysoká dostupnost takového řešení je další výhodou cloudu. Poskytovatel služby má spoustu odborníků na danou problematiku, pravidelně zálohuje data, obnovuje hardware a celkově se stará o chod a vylepšování služeb, které nabízí. Cloud computing při všech těchto výhodách není pouze pro velké a bohaté společnosti. Firmy přecházející do prostředí cloudu nemusí řešit velké počáteční investice. Když se rozhodnou, že chtějí vybranou službu začít využívat mohou začít téměř ihned, stejně tak jako přestat službu využívat ze dne na den.\par
    Tyto prvky cloud computingu jsou možné díky technologiím jako je virtualizace, automatizace a orchestrace zdrojů a řešení pro vysokou dostupnost. Virtualizace je technika, která umožňuje běh více virtuálních serverů na jednom fyzickém serveru. \newline Fyzický server emuluje pro každý virtuální server hardware, to znamená procesor, RAM paměť, síťovou kartu a další. Uživatel vnímá virtuální server jako hardwarový server a nepozná rozdíl. Přitom jednotlivé virtuální servery jsou od sebe navzájem izolované. Virtualizace umožňuje lepší využití zdrojů a automatizaci vytváření virtuálních serverů. Virtualizace se ale netýká pouze virtuálních serverů, ale také počítačových sítí a síťových prvků a také datového uložiště. Jednotlivé koncepty jsou představeny v  \cite{murphy2017virtualization}. \par
    Aby mohli uživatelé dynamicky pracovat se zdroji, je potřebné automatizovat správu jednotlivých zdrojů. Každý poskytovatel cloudového řešení představuje vlastní přístup k tomuto problému. Uživatelé mohou poté přes webový portál či API dynamicky spravovat zdroje bez zásahu poskytovatele cloudového řešení. Přidání, změna či \linebreak odebrání zdrojů se děje bez zásahu poskytovatele. Jednotlivý uživatelé jsou účtováni na základě využívání jednotlivých zdrojů. 

\section{Dělení cloud computingu}
    Cloud computing můžeme rozdělit do tří kategorií podle služeb, která má daný model poskytovat. Zmíněnými modely jsou Infrasracture as a service (IaaS), Platform as a service (PaaS) a Software as a service (SaaS). Dalším aspektem podle kterého může být cloud computing dělen je způsob nasazení daného servisního modelu. Servisní modely určují jaké služby budou využívány. Tyto modely se mohou dále lišit podle způsobu nasazení. Zde jsou 3 základní modely nasazení. Private cloud vlastněný organizací, public cloud sdílený více organizacemi a hybrid cloud kombinující private a public cloud. \par
        Takovéto dělení cloud computingu bylo relevantní v době svého vzniku. Dnes se jednotlivé rozdíly smazávají, dělení není tak striktní a trendem je propojení více \linebreak modelů dohromady tak, aby uživatelům cloudu nabídlo výhody každého z řešení. Příkladem může být například banka HSBC. Tato banka se snaží integrovat \linebreak nejmodernější technologie, které jsou dostupné tak aby z toho měla co největší přidanou hodnotu pro svoje zákazníky. Tato společnost využívá AWS (Amazon web services) cloud. Amazon například pomohl s přesunem dat z privátního datacentra. Toto spojení pomáhá HSBC nasadit aplikace v rámci sekund místo měsíců. IaaS a PaaS \linebreak modely také ztrácí na svém významu. Např. HSBC používá AWS Lambda službu. Tato služba abstrahuje uživatele od správy serverů a zaměřuje se pouze na vykonání funkce. Uživatel tak pouze pošle data do Lambdy a zpět obdrží vypočtená data nebo relevantní odpověď. Takovéto řešení velice dobře škáluje a umožňuje souběžný výpočet pro velké množství transakcí \cite{devopsonline}. Dalším aspektem je využití multi cloud řešení. Využítí více jak jednoho cloudového řešení. Příkladem může být opět banka HSBC, která využívá kombinaci AWS a GCP (Google Cloud platform). GCP nabízí nástroje pro analýzu dat a strojové učení, které HSBC používá pro podporu vnitřních rozhodovacích procesů a analýzu velkého objemu dat \cite{digitalnewasia}. Z toho můžeme vyvodit, že dnes společnosti využívají výhody jednotlivých poskytovatelů služeb a kombinují je tak, aby bylo dosaženo co nejefektivnějšího řešení, které bude dané společnosti vyhovovat, protože jeden poskytovatel nepokryje všechny specifika. Společnosti tak mají svobodu ve výběru technologií od různých poskytovatelů cloudových služeb.\par
	Dalším důvodem pro využití multi cloud řešení je i legislativa, nezávislost pouze na jednom poskytovateli služby a dostupnost aplikací. Legislativa například omezuje \linebreak v\,jaké geografické lokaci mohou být data občanů dané země uloženy. Toto nařízení se týká především nadnárodních společností, které působí po celém světě. Například \linebreak podle přehledu v tabulce \ref{table:1} vidíme jednotlivé regiony cloudových poskytovatelů AWS, Azure, Google cloudu a Alibaba cloud. Z tabulky je patrné, že každý poskytovatel má své datové centrum ve velkých evropských zemích jakou jsou Německo, Francie a Velká Británie. Dále každý poskytovatel přidává lokaci v menší zemi. AWS má datové centrum ve Švédsku, Azure ve Švýcarsku a Google cloud ve Finsku. Vyjímkou je Alibaba cloud. Tento čínský poskytovatel cílí hlavně na asijský trh a v evropě tak zatím nemá tak velké zastoupení. Multi cloud řešení minimalizuje závislost na jednom poskytovateli. Je to ochrana před radikální změnou politiky daného poskytovatele, případně zmenšuje škodu způsobenou výpadkem služeb daného poskytovatele.

\begin{table}[h]
\begin{tabular}{c|c|c|c}
\hline
	\textbf{AWS}    & \textbf{AZURE}   & \textbf{GCP}    & \textbf{Alibaba cloud}  \\ \hline
	Irsko  & Německo & Londýn               & Londýn         \\ \hline
	Londýn & Francie & Hamina, Finsko       & Asie           \\ \hline
	Paříž  & Londýn  & Emshaven, Nizozemsko &                \\ \hline
	Stockholm, Švédsko & Nizozemsko & Belgie &               \\ \hline
	       & Švýcarsko &            &       &               
\end{tabular}
\caption{Tabulka lokací datových center různých poskytovatelů}
\label{table:1}
\end{table}
