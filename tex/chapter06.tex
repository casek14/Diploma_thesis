\chapter{Návrh a implementace systému pro běh distribuovaných aplikací}
\section{Definice požadavků na běh distribuované aplikace}
V předchozí kapitole jsou představeny projekty, které řeší otázku jak spravovat jednotlivá cloudová řešení a aplikace napříč různými cloudovými poskytovateli co nejefektivněji z jednoho místa. Současným trendem jak provozovat aplikace v prostředí cloudu je využití kontejnerizace a nástroje Kubernetes pro správu těchto kontejnerů, viz. kapitola Cloud native. Kubernetes architektura a nástroje pro správu K8s jsou navrženy pouze pro provozování jednoho clusteru. Mnoho společností se zajímá o propojení jednotlivých clusterů tak aby tvořily jeden celek. Tento přístup přináší mnoho výhod. Uživatelé mají možnost spravovat několik clusterů z jednoho místa. Z tohoto centrálního uzlu by uživatelé měli být schopni vytvářet zdroje v jednotlivých clusterech a také odtud sledovat chování a stav jednotlivých clusterů. Další výhodou je využití několika poskytovatelů pro lepší dostupnost svých služeb. Při výpadku jednoho poskytovatele máme pořád běžící aplikaci u druhého poskytovatele. Dalším použitím více clusterů je Edge cloud koncept. K8s clustery, umístěné v místě potřeby snižují odezvu na události a zvyšují dostupnost pro kritické aplikace. Propojením jednotlivých k8s clusterů se zabývá spousta společností a také k8s komunita. Konkrétně se jedná o projekt Kubernetes Federation, který je představen v předchozí kapitole.
     Správa více k8s clusterů je komplexní úkol a musí splňovat několik požadavků, aby bylo možné jednoduše a z jednoho místa efektivně provozovat distribuovanou aplikaci: 
\begin{itemize}
\item První požadavkem takového systému je možnost jednotně spravovat desítky až tisíce k8s clusterů. Systém by měl nabízet jeden vstupní bod pro obsluhu všech k8s clusterů z jednoho místa.  Manuálně spravovat jednotlivé clustery není \linebreak výhodné, protože toto řešení špatně škáluje. Pokud se například společnost, \linebreak využívající edge cloud v každé pobočce, bude rozrůstat o nové pobočky, bude muset pro každé dvě až čtyři pobočky zaměstnat nového operátora k8s clusteru. Efektivnějším řešením je definovat aplikaci v centrálním systému a poté podle potřeby distribuovat danou aplikaci na jednotlivé clustery. Výhodou použití k8s v jednotlivých lokacích je stejné a konzistentní prostředí přes všechny clustery. Odpadá tak nutnost ovládat technologie od různých společností. Spouštění a provozování aplikací bude v jednotlivých lokacích probíhat stejně jako v testovacím prostředí. Pro doručování nových verzí aplikace, nabízí k8s tkz. rolling updates. Tato funkce umožňuje bezvýpadkové nahrání nové verze aplikace. K8s postupně nahrazuje pody se starou verzí aplikace, jeden po druhém a zároveň spouští pody s novou verzí. Nové funkce a opravy tak mohou být rychleji aplikovány. Pokud dojde k nahrání špatné verze aplikace, která nebude fungovat, k8s umožní se rychle vrátit na předchozí funkční verzi. 
\item     Dalším plusem je použití k8s architektury a všech zdrojů a technik, které k8s nabízí. K8s je robustní nástroj, který v sobě zahrnuje vysokou dostupnost, škálovatelnost, rozšiřitelnost, je to řešení odolné vůči chybám, vyvíjené, spravované a podporované širokou komunitou. Pro zmíněné atributy se k8s stalo de facto standardem pro orchestraci kontejnerů. Například použití Pod konceptu, namísto kontejneru. Pod dovoluje seskupit více kontejnerů, které na sobě úzce závisí a potřebují být spuštěny na stejném serveru nebo mezi sebou sdílejí zdroje, do jednoho logického celku. Pod dovoluje také využití tkz. Side-car konceptu. Side-car je označení pro kontejner, který běží ve stejném podu jako kontejner uživatelské aplikace a poskytuje další služby jako je sběr logů z uživatelského kontejneru a jejich odeslání na centrální server. 
\item Pro přístup více subjektů do systému je nutné izolovat zdroje jednotlivých subjektů tak, aby na sebe vzájemně neviděly a nemohly tak ovlivňovat aplikace ostatních zákazníků. Pro tento problém poskytuje k8s namespace. Namespace je virtuální cluster uvnitř fyzického clusteru. Jednotlivé zdroje uvnitř namespace jsou izolovány od ostatních. Uživatelé jsou schopni vytvářet, prohlížet, upravovat a mazat pouze zdroje v přiděleném namespacu. Namespace může být využit pro oddělení jednotlivých aplikací, různá oddělení jedné společnost mohou mít vlastní namespace, případně různé firmy mohou být odděleny pomocí namespacu. 
\item Dalším požadavkem je spuštění různých aplikací v clusterech situovaných \linebreak v různých lokacích. Např. aplikace, kterou zákazníci v obchodě používají pro výběr zboží budeme provozovat ve všech obchodech. Aplikaci pro interní zaměstnance, která ulehčuje správu skladů se zbožím budeme provozovat ve všech skladech a také v největších obchodech s vlastním skladem. Pro tento učel je možné využít labely, neboli popisky jednotlivých objektů. Labels v k8s mají \linebreak podobu klíč a k němu přiřazená hodnota. Jednotlivé k8s deploymenty je možné popsat labely clusterů, ve kterých má být daný deployment spuštěný. Aplikace pro zákazníky by tak mohla obsahovat label lokace:praha a label typ:prodejna. Pomocí labelů je tak možné jednoznačně určit lokace a typ prodejny na které má být aplikace spuštěna. 
\item Důležitým aspektem pro běh distribuovaných aplikací je řízení přístupu uživatelů k jednotlivým zdrojům. K8s obsahuje několik mechanismů pro ověření zda uživatel má právo provést danou akci či nikoliv. Administrátoři tak mohou kontrolovat jaké akce mohou uživatelé vykonávat. Více o problematice a typech k8s autorizace je uvedeno v \cite{k8sauth}.
\end{itemize}
